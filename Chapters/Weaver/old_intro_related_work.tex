\section{Introduction}

Whether planning a trip to a new city, figuring out which camera to purchase, or researching the different treatments for a medical issue, people spend a significant amount of time gathering evidence across multiple webpages to make decisions. Unlike simple searches that may require finding only a single trusted source, such as the address of a restaurant or checking the weather, complex exploratory search tasks often involve performing multiple searches and require significant amounts of learning, cross-referencing, and synthesis across multiple webpages. Consider for example planning a trip: there may be hundreds of possible restaurants to dine at, attractions to see, and places to stay, each with corresponding evidence about its suitability for an individual's goals. Evidence about each of these entities is often spread out across many different pages, including Yelp or TripAdvisor reviews, travel blogs, discussion forums, top ten lists, and travel guides. Furthermore, each piece of evidence on its own may be susceptible to its creators' biases, context, and opinion, requiring users to synthesize across many pieces of evidence and entities in order to build a reliable landscape of the decision space. Research estimates of the amount of time spent gathering and synthesizing evidence for such complex search tasks to be up to 33\% of the time spent online, which amounts to a total of more than 24 billion hours per year in the US alone (as of 2009) \cite{mar2006exp,kellar2007field,rose2004understanding,forrester}.

Significant research has gone into supporting various aspects of the above activities. One major thread of research has been around knowledge graphs composed of entities and how they relate to each other \cite{dbpedia}. These entities can then be surfaced to the user, typically by search engines alongside search results. In these systems, entities are presented as a card or a list of names, often with corresponding images \cite{bota}. While such approaches focus on and can be effective in helping users discover what entities are related to a particular information need (e.g., \cite{miliaraki2015selena}), simply knowing about the existence of an entity may only be the beginning of users' information needs. Without an understanding of context -- which web sources mention an entity, how it is mentioned across those sources, what aspects of those mentions are relevant to users' goals, and how it compares to alternative entities -- users might not know which entities are trustworthy and relevant to their goals.

To understand the prevalence of this problem, we conducted a pilot survey with 103 participants from Amazon Mechanical Turk (age between 20 and 63, $\bar{x}$=36, $\sigma$=12, 60\% male and 40\% female, mostly from the US), focusing on their experiences when conducting complex exploratory searches. Our results reaffirm that when conducting exploratory searches, people read from multiple information sources to make sure they do not miss something important (84\%), look for more information about a topic from other webpages in the same search result (84\%), verify previously seen but uncertain information using multiple sources (88\%), and get a sense of what's popular or important in the search results (76\%). At the same time, many also found that conducting complex searches can be stressful (60\%), and that they often end up with more browser tabs than they can manage efficiently (67\%).

Another thread of addressing the problem has been tools for helping users gather and save snippets of evidence across many pages. Such tools have become popular and widespread in recent years, including Evernote with over 200 million users and Pocket with over 2 billion items saved. Researchers have also explored ways to support saving information on the web, reducing the costs and extending the capabilities of such tools, for example support saving structured data \cite{thresher,bier2006entity}. However, such systems have typically focused on creating separate environments to save and organize the collected snippets.  Accessing a separate system for context every time users encounter an item can potentially incur significant challenges, with the costs of reusing items or removing unused items in that system becoming prohibitive \cite{notetoself}.

In this paper we explore an alternative approach that scaffolds users' current browsing experience to help them gather and synthesize evidence about entities across multiple webpages. The key insight we build on is that while users aim to collect evidence for entities across multiple sources, their browsers currently treat each of the instances of those entities as independent. If we instead understood that the same entity was being mentioned on multiple different pages, we could develop novel systems that can better support users in understanding the contexts that entity was mentioned in and how those contexts relate to their personalized needs and goals. Here we introduce one such system, Fusion, which instantiates this approach in a prototype browser addon. Fusion analyzes and identifies entities mentioned in individual pages, helping the user gradually build up a workspace of options and evidence that persists across webpages throughout the task. Using an entity-centric approach has the potential to not only reduce the costs of keeping track of options and gathering evidence, but also enables new end-user interface approaches, such as proactively showing a user how many sources mention an entity and the contexts in which it is mentioned, or propagating a users' notes and annotations across all pages an entity is cited.

The core contribution of this paper is an entity-centric approach for supporting sensemaking across webpages in the browser when conducting complex exploratory search tasks, which includes the following:

\begin{itemize}
\item Fusion, a browser addon that enables the browser to better understand the content of webpages by identifying entities using existing natural language processing algorithms and external knowledge sources (e.g., Wikipedia and Yelp).
\item Support for reading and gathering information during exploratory search tasks by lowering the cost of cross-referencing different entities across webpages and searches
\item A context-aware workspace where users can easily gather evidence from multiple webpages for a collection of entities which propagates to other pages that mentions the same entities. Allowing the system to surface relevant  information saved previously as users browse different pages.
\end{itemize}


\section{Related Work}

\subsection{Search Engines and Exploratory Search}

Research on interfaces that support exploratory search tasks have focused on providing overviews of information within search results or managing multiple searches, so searchers can better orient themselves in an unfamiliar information space \cite{hearst2009search,marchionini2000agileviews,patterson2001predicting,tretter2013searchpanel,morris2008searchbar}. Recent research has also explore ways to support managing multiple search sessions and project using novel browser interfaces \cite{hahn2018bento}. On the other end of the process, recent researchers have also explored ways to support learning by enriching and personalizing  search results when users conduct exploratory searches \cite{syed2017optimizing}. Alternatively, studies have also shown the benefits for surfacing related entities from external knowledge bases on the search results page for quick reference \cite{bota,miliaraki2015selena}, forming subsequent query terms \cite{klouche2015designing}, or collaboration \cite{andolina2018querytogether,klouche2018hyperlinks}. However, ranking algorithms and interactive search result interfaces provide little support once users opened webpages from the results to read and collect information from. At this stage of the process, not only are the individual webpages disconnected from the search results once opened, these webpages typically contain overlapping information around the same topics but are also treated as independent of each other, requiring users to manually identify common themes in different sources and collect evidence for synthesis.

While Fusion also identifies relevant entities and presents users with information from external knowledge bases, it uses this information to support lightweight cross-reference between webpages  in their search results while reading individual webpages without opening multiple tabs and switching between them, and also to support lightweight cross-referencing of users' own notes saved from different pages.

\subsection{Saving and Organizing Information}

When reading the individual pages, users in exploratory search tasks often need to take notes or save information from different webpages. Due to its ubiquity and importance, researchers have also been developing systems that can better support saving and organizing information to facilitate learning and exploratory search tasks \cite{schilit1998beyond, tashman2011liquidtext,hinckley2012informal,kittur2013costs,notetoself,chang2016supporting}. 
However, researchers have also identified common issues users are faced with while organizing collected information. Karger et al. pointed to the fact that most information management systems failed to provide effective structure for its users due to the long-tail distribution of information types that people have \cite{whatwentwrong}. Further, Kittur et al. found that in early stages of exploration, often searchers themselves do not have enough context to come up with effective structures \cite{kittur2013costs}. These studies revealed systems that force schematization too early in the exploration process stages to can be detrimental to their users. In a study closely related to our work, the NoteToSelf system added a sidebar to the browser for saving free-form notes that persist as users browse different webpages \cite{notetoself}. However, they found that while participants created many notes during browsing, they rarely revisited these notes nor deleted unused notes. This suggests that even though allowing users to externalize their arbitrary structures can be beneficial, but the high cost of recalling and re-finding previously saved information can still be prohibitive.

In our work, we address the information reuse problem using an entity-centric approach. As users opened a new webpage, Fusion analyzes the content and resurfaces previously saved notes that were associated with entities also on the current webpage in real-time. Similar to \cite{notetoself}, we also used a lightweight interface that is integrated into users' reading process, and allowed them to either type notes or extract content from different webpages. In our study, users expressed how this allowed them to quickly refer back to what they have learned in the past, and also reduced the cost of removing previously saved information that were no longer relevant. 

\subsection{Identifying Entity Mentions in Web Content}

There has been significant previous work in identifying entities from unstructured or semi-structured webpages. Approaches include wrapper induction techniques based on user interaction \cite{thresher}, which typically rely on multiple pages with a common structure or data format \cite{pasupat2014zero}; natural language patterns (e.g., \cite{hearst1992automatic} and \cite{fader2011identifying}) which use seed examples; or exploit regularities in semi-structured data \cite{thresher}. Our work builds on these advances in entity extraction by using a state-of-the-art entity linking algorithms for the DBpedia knowledge base \cite{spotlight,dbpedia}, and a custom algorithm for disambiguating entities across DBpedia and Yelp. However, as pointed out by Klouche \cite{klouche2018hyperlinks}, while extant approaches to entity extraction have mainly focused on search engine or question answering outcomes, there is a largely unexplored design space around using entities to drive novel end-user interactions. Our work explores one such paradigm in which entities are used to scaffold the gathering and synthesis in a user's online sensemaking process. Similar to \cite{thresher}, we also allowed users to manually extract entities from webpages, but mainly as a way to recover from errors made by the automatic entity linking algorithm \cite{spotlight}.

