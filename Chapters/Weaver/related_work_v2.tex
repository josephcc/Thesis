\subsection{Sensemaking and the Web}

The importance of sensemaking and complex exploratory search on the web has been studied in depth by many researchers. Past work have identified a persistent challenge that valuable information for many topics is scattered across many different sources that are independent of one another and incur a high cost for bringing them together \cite{bhavnani2005difficult,mar2006exp,marshall1999introducing,kittur2013costs}. Theories of sensemaking suggest several cognitive tasks involved that could be supported through novel interactive tools, ranging from finding potentially relevant items, to triaging items based on reliability and relevance, to collecting evidence relevant to each item and organizing items into categories or structures \cite{russell1993cost, takayama2008tracing, hearst2013sewing}. We draw on these theories to select the set of cognitive tasks we are interested in supporting through entity-centric interaction approaches, which are typically complex and about synthesizing information, rather than simple fact-finding tasks \cite{mar2006exp, white2006supporting}.

\subsection{Recognizing Entities in Text}

Significant research has gone into entity-centric approaches for improving web search results pages due to the ubiquity of entities in online sensemaking. Researchers have found that entity-bearing and -category queries accounted for up to 85\% of web search traffic \cite{guo2009named,lin2012active}. This has led to significant academic and commercial efforts devoted to building large-scale entity databases (such as DBPedia \cite{dbpedia}, Yelp\footnote{http://yelp.com}, and Google Places\footnote{https://developers.google.com/places/web-service/intro}), and significant research on ways to identify entity mentions in plain text \cite{spotlight} and using them to enrich search interfaces. This involves both recognizing the same entity mentioned in different surface forms (e.g., MoMA and Museum of Modern Art) and resolving ambiguous surface forms to the right entity based on its surrounding text. Major threads of research that uses entities to improve search interfaces includes showing entity cards for entity-bearing queries \cite{bota,miliaraki2015selena}, answer factual questions  \cite{D15-1038}, and showing related entities as query suggestions \cite{blanco2013entity, bordino2013penguins,klouche2015designing}. We build on these recent advances in entity recognition and large-scale entity databases, but instead of focusing on the search interface and simple navigation we investigate the less-explored design space of supporting complex sensemaking across webpages opened in the browser through entity-aware interactions.


\subsection{Weaving Together Scattered Information across Sources}

Due to the scattered nature of information on the Web \cite{bhavnani2005difficult}, research has explored ways to connect relevant information distributed across different webpages. One set of top-down semantic web approaches involve incentivizing content publishers to provide machine readable annotations, such as using semantic web markups \cite{karger2004haystack, berners2001publishing,berners2001weaving}. However, such approaches have often failed to gain momentum due to issues such as a lack of available end-user tools that can consume these annotations \cite{whatwentwrong}. Alternatively, researchers have built bottom-up systems that exploit detecting entity mentions in articles, and used them as anchor points to connect to other information sources to enhance the reading experience. For example, Wikify identifies entity keywords in articles and creates hyperlinks to their Wikipedia entries for navigation \cite{mihalcea2007wikify}, and Experience-Infused
Browser links entity mentions in articles to past social interactions (such as emails) for making ``serendipitous connections'' \cite{hangal2012effective}.

Our approach is inspired by these bottom-up approaches in the context of recognizing entities mentioned in webpages opened in the browser. However, our work differs in several important ways: 1) instead of surfacing simple facts or links to articles or emails, we focus on providing context for people to understand complex information spaces; 2) we support the transition from viewing context to saving information; and 3) we propagate saved information to all other instances that entity is shown, allowing users to build up an entity-based mental model of the space.

%Instead of navigating users to a separate page, we used an in situ design that allowed users to hover over an entity mention to bring out its entity card \emph{infused} with relevant information extracted from other information sources. In addition to information from external knowledge sources (such as Wikipedia), we also used the entity cards to show relevant information extracted from other opened webpages that also mentioedn the same entity for lightweight cross-referencing.


\subsection{Note Taking and Saving Information Online}

Collecting information online during complex sensemaking tasks can be costly for the users, requiring them to cross-reference between pages and their notes in order to gather all the evidence. This frequent context switching between different documents and taking notes can be distracting, and even prohibitive for users to investigate deeper or to take notes in order to avoid disrupting the flow of reading \cite{o1996towards,marshall1999introducing,tashman2011liquidtext,bianchi2015designing}.

On one end of the spectrum, research has focused on allowing users to extract and save structured information from webpages more efficiently from a single document using end-user programming \cite{thresher,huynh2005piggy,dontcheva2006collecting,dontcheva2007relations} and interaction techniques \cite{bier2006entity,stylos2004citrine}. Our work is motivated by these studies highlighting users' desire to collect information about entities, but instead of focusing on structured and objective attributes we are interested in additionally gathering descriptive and subjective evidence, such as how a restaurant is described in a top ten list. Furthermore, many of the approaches above define patterns for extracting multiple entities from a single page, whereas we are more interested in finding evidence related to entities across multiple pages where the structure of those pages might differ significantly.

On the other end of the spectrum, researchers have also explored using in situ interfaces, such as sidebars or on page annotations, to reduce the costs of switching between reading and note taking and collecting from multiple information sources \cite{romat2019spaceink,tashman2011liquidtext,notetoself,van2011finders,schraefel2002hunter}. Our work is also situated in this thread of research on reducing the costs of switching and sensemaking. However, instead of persisting notes on individual documents, our approach persists notes on individual entities which may then appear across multiple documents. 

%Unstructured note taking approaches also run into challenges with scaling for reusing and organizing large collections of saved notes.  For example, a prior system, NoteToSelf, reported that their participants relied on skimming or targeted search using keywords from memory for re-finding previously saved notes, which led to a large portion of user notes rarely re-accessed nor deleted \cite{notetoself}. Fundamentally, note taking software treats options mentioned on webpages and users' notes about them as independent, and cross-referencing between webpages and their notes to accumulate evidence for the same options can be incur high costs. In this work we explore an entity-centric approach that connects different webpages and users' notes through common entity mentions, allowing us to automatically resurface previously saved notes as they became relevant to users' current reading.


