
\todo{today entities primarily used for search. but there is an opportunity and history of entities in sensemaking too, e.g., entity workspace, etc. but this has been very heavyweight and manual.

we are inspired by lightweight in situ approaches such as notetoself -- can we take this type of approach and apply to entities?  

or

on one end of spectrum is super lightweight unstructured stuff like notetoself — no entities, but very lightweight.  on other end is very heavy structured stuff like entity workspace — use of entities, but very heavy as everything is manual.  we posit that with the recent developments in entity recognition there is an opportunity for lighter weight use of entities and a design space around enabling user interactions for sensemaking through entity awareness.}

Our approach draws on previous work in sensemaking and entity-centric approaches, and brings them together to propose and explore a design space for interactions that help users connect information scattered across many sources on the internet. Below we discuss the main threads of research that we build on.

\subsection{Sensemaking and the web}
The importance of sensemaking and complex exploratory search on the web has been studied in depth by many researchers, who have identified a persistent challenge that valuable information for many topics is scattered across many different sources that are independent of one another and incur a high cost for bringing together (\cite{bhavnani,mar2006exp,marshall1999introducing,kittur2013costs}. Theories of sensemaking suggest several cognitive tasks involved that could be supported through novel interactive tools, ranging from finding potentially relevant items, to triaging items based on reliability and relevance, to collecting evidence relevant to each item and organizing items into categories or structures \cite{russell1993, takayamapirollicard, hearst2013sewing}. We draw on these theories to select the set of cognitive tasks we are interested in supporting through entity-centric interaction approaches, which are typically complex and about synthesizing information, rather than simple fact-finding tasks \cite{mar2006exp, white2006supporting}.

Researchers have also built tools to support online foraging from saving and organizing entire webpages \cite{card1996webbook} to specifying and saving parts of webpages and organizing them \cite{dontcheva2006collecting, dontcheva2007relations,sugiura1998internet,zhu2002hunter,chang2016supporting}.

There have been a variety of tools developed to support sensemaking on the web, including WebBook \cite{card1996webbook}, Hunter Gatherer \cite{zhu2002hunter}, Haystack \cite{karger2004haystack}, and several dozen others that address some aspect of foraging or building representations of web content. Most relevant to this work is research that helps users collect information about specific topics from scattered information across the web, and to annotate and propagate annotations about those topics.

%Past research has proposed a variety of approaches to better support complex search tasks at varying stages. This includes in situ interfaces for note-taking in the browser, interaction techniques for extract structured entity attributes, and presenting rich entity cards in search results. Our work builds on this diverse literature by leveraging state-of-the-art entity recognition algorithms in the browser to drive our Infusion- and Diffusion-based interactions. This allowed us to empower the browser interface to better understand the information being consumed by its users, and use an entity-centric approach to support sensemaking and information foraging across multiple webpages.

%\subsection{Saving and Organizing Information from the Web}

%Users often need to take notes or save information from webpages when searching online \cite{mar2006exp, lee2012survey}. Due to its ubiquity, browser add-ons supporting online foraging and organizing information have become popular and widespread in recent years, including Evernote with over 200 million users and Pocket with over 2 billion items saved. Researchers have also built tools to support online foraging from saving and organizing entire webpages \cite{card1996webbook} to specifying and saving parts of webpages and organizing them \cite{dontcheva2006collecting, dontcheva2007relations,sugiura1998internet,zhu2002hunter,chang2016supporting}.

%However, saving information from webpages during complex exploratory search tasks, such as trip planning or shopping, can still be costly for the users. In these scenarios users are often faced with many entity options and rich attributes (such as specification of different cameras), and the majority of this information is in unstructured plain text requiring significant effort to extract. One approach requiring content publishers to provide machine readable annotations such as using semantic web markups \cite{berners2001publishing,berners2001weaving}, has failed to gain momentum due to a lack of available end-user tools that can consume these annotations \cite{whatwentwrong}. In addition to collecting structured information, subjective and descriptive information for each option are often scattered across many information sources (such as reviews and top-10 lists), requiring users to cross-reference between pages and their notes in order to gather all the evidence. This frequent context switching between different documents and taking notes can be distracting, and even prohibitive for users to investigate deeper or to take notes in order to avoid disrupting the flow of reading \cite{o1996towards,marshall1999introducing,tashman2011liquidtext,bianchi2015designing}.

%Extracting and saving information across webpages in an exploratory search task can be challenging in that these webpages typically contain evidence for overlapping options, but the majority of the information is unstructured, requiring users to manually cross-reference between pages in order to gather evidence about the same option.

At a high level, this work spans a spectrum from efforts focusing on collecting and structuring information into connected entities, to lightweight ways of taking notes and collecting information about topics without interrupting the user flow. At one end, researchers have explored ways to help users to collect entities and extract their structured attributes more efficiently using end-user programming and interaction techniques. For example, \cite{bier2006entity} and \cite{stylos2004citrine} allowed for efficient copying and pasting of entity attributes  (e.g., addresses and phone numbers) by automatically identifying them in text, and \cite{thresher,huynh2005piggy,dontcheva2006collecting} assisted users in extracting both entities and their attributes from webpages (e.g., a list of faculty with contact information). Our work builds on this studies which highlight users' desire to collect collect information about entities, but instead of focusing on structured and objective attributes we are interested in additionally gathering descriptive and subjective evidence. Furthermore, many of the approaches above define patterns for extracting multiple entities from a single page, whereas we are more interested in finding evidence related to entities across pages (and where the structure of those pages might differ significantly).

On the other end of the spectrum, researchers have also explored using in situ interfaces (e.g., a sidebar) to reduce the costs of switching between reading and note taking and collecting from multiple information sources \cite{romat2019spaceink,tashman2011liquidtext,notetoself,van2011finders}. We draw on the findings of these systems that reducing costs of switching and saving information can be important to supporting sensemaking. However, while these systems primarily focus on persisting notes on individual documents we are interested in synthesizing across sources. Unstructured note taking approaches also run into challenges with scaling for reusing and organizing large collections of saved notes.  For example, \cite{notetoself} reported that their participants relied on skimming or targeted search using keywords from memory for re-finding previously saved notes, which led to a large portion of user notes rarely re-accessed nor deleted. Fundamentally, note taking software treats options mentioned on webpages and users' notes about them as independent, and cross-referencing between webpages and their notes to accumulate evidence for the same options can be incur high costs. In this work we explore an entity-centric approach that connects different webpages and users' notes through common entity mentions, allowing us to automatically resurface previously saved notes as they became relevant to users' current reading.

In a study more closely related to our work, \cite{dontcheva2007relations} allowed users to both manually create entity cards by extracting structured information from one webpages, and also use them as a placeholder to collect more structured information from other webpages. However, due to a lack of support for automatically recognizing entities in text, users still had to manually cross-reference between entities mentioned on different webpages and in their collection of cards in a separate workspace in order to save more information. To address this, we took an approach similar to \cite{hangal2012effective} where entity mentions on a webpage are automatically recognized and highlighted. In their system, users can click on an entity mention to bring out past emails and instant message logs that also mentioned the same entity. While \cite{hangal2012effective} focused on supporting making ``serendipitous connections'' between the current readings and past social interactions, we focused on supporting sensemaking across information sources during complex exploratory search tasks. Finally, both systems were tested with limited numbers of participants (6 and 7, respectively). To better understand the costs and benefits of using an entity-centric approach in the browser, we compared our approach to a baseline system similar to \cite{notetoself} with 24 participants as a between subject condition, and report both behavior statistics and interview responses.

In this paper, we built a prototype browser add-on, {\SYSTEM}, that aims to address the aforementioned limitations of prior work --- the high costs of reusing previously saved notes and evidence, cross-referencing, and context switching. {\SYSTEM} utilizes open and commercial entity databases \cite{dbpedia} and state-of-the-art entity linking algorithms \cite{spotlight} to automatically identify entities mentioned across information sources in an exploratory search task. When a user encounters an entity on a webpage, {\SYSTEM} presents an in situ entity card with rich attributes similar to ones used in modern commercial search interfaces \cite{miliaraki2015selena,bota}. For example, showing the location and review ratings of different restaurants mentioned on a webpage. The in situ entity cards also serve as a foraging structure where users can attach notes and web clips to them. This also allows {\SYSTEM} to automatically resurface previously saved notes and clips on other webpages that also mention the same entities so previously saved information can be efficiently reused.

\subsection{Entities in Search Interfaces}

Users conducting exploratory search tasks are unsure about their goals, and often need to rely on reading and foraging from multiple webpages to iteratively learn about the available options and gather useful evidence \cite{mar2006exp}. Prior work on search engine interfaces have focused on ways to help searchers better orient themselves by providing an overview of webpages in a search result \cite{marchionini2000agileviews,patterson2001predicting,tretter2013searchpanel} or managing multiple searches and information sources \cite{morris2008searchbar,hahn2018bento}. More closely related to our work, significant research has also gone into entity-centric approaches due to the ubiquity of entities in online sensemaking tasks. Studies in 2009 and 2012 have found that entity-bearing queries and entity category queries accounted for up to 71\% to 85\% of web search traffic \cite{guo2009named,lin2012active}. This has led to significant academic and commercial efforts devoted to building large-scale entity databases (such as DBPedia \cite{dbpedia}, Yelp.com, and Google Places), and a decade of research on ways to enrich search interfaces with information about entities. Major threads of research include identifying entities mentioned in queries to present entity cards for quick referencing \cite{bota,miliaraki2015selena}, answering factual questions about entities directly \cite{D15-1038}, and showing lists of related entities to be used as subsequent queries \cite{blanco2013entity, bordino2013penguins,klouche2015designing}. 

While these approaches have made great strides in making retrieval more efficient based on better understandings of users' intent \cite{miliaraki2015selena} and the content of web documents \cite{fernandez2008semantic}, significant user effort is still required after retrieving documents --- when consuming and extracting information from individual webpages and organizing them. Yet the browser interface where users conduct these intense sensemaking and decision-making processes remain largely unchanged, and relatively less explored in research \cite{whatwentwrong}. Current browsers treat entity mentions and their evidence described in each opened webpage independent of other webpages, making it difficult for users to cross-reference, forage, and keep track of what they are interested in and why across webpages. In this paper, we instead utilize techniques used in search systems to explore an alternative design space where the browser is aware of the same entities mentioned across webpages opened during an exploratory search task. For this, we propose to empower the browser interface with readily available open and commercial entity databases \cite{dbpedia} and entity linking algorithms \cite{spotlight}, and explore entity-centric approaches for supporting sensemaking across webpages.