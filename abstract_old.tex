

Foraging and consuming information online has became how many people make sense of the world today. In order to make informed decisions, people spend hours, even days, using search engines to retrieve lists of webpages, and gather information from them to plan trips, figure out what to purchase and where to eat, and learn new skills.

However, many of these complex exploratory searches requires significant amounts of learning and synthesis across multiple online sources -- in order to make informed decisions, users need to discover, gather, structure, and evaluate potential options and their supporting evidence scattered across webpages and external note-taking softwares, making it difficult for users to keep track of what they are interested in and why.


- discovering options - what are all the options? how do they compare? do these 3 articles cover everything?
- whats some good qualities of a good ramen restaurant? what terms can describe these quality better?
- would i like this option? is this for me? (based on my own preferences / scenarios)
- what do i need to know? is this going to be relevant in the future?
- have i seen this before? whats relevant on this page? what did other sources say about this options?
- are there other sources i can use for this option?


My thesis work explores ways to better support this exploratory search and information foraging process under two contexts:
1) \textbf{Providing Overview}: Synthesize information scattered across search results into overview articles (\cref{chap:alloy,chap:ka}) or interactive search result pages (\cref{chap:searchscape}) by using  crowdsourcing and machine learning techniques to identify useful concepts or options.
2) \textbf{Supporting Foraging}: Allow users to express and iteratively refine rich expressions of personalized interests and criteria during foraging, and support saving information under uncertain scenarios with low effort (\cref{chap:highlight}).



In the first half of the thesis, I focus on the query stage of the exploratory search process, using crowdsourcing and machine learning techniques to discover important concepts from webpages in search results, and use them to organize web content into useful structures (\cref{chap:alloy,chap:revolt}) that can be further synthesized into how-to styled reports (\cref{chap:ka}). Alternatively, build interactive search interfaces (\cref{chap:searchscape,chap:searchlens}). In the second half of the thesis, I focus on building novel systems and interaction techniques (\cref{chap:highlight}) that can better support for collecting and organizing information while users were reading from multiple sources.

Building on my prior work and a preliminary study (\cref{chap:tabs}), I propose to explore ways to better support online information foraging by automatically associating relevant information scattered across multiple webpages. To test this idea, I plan to develop an intelligent note-taking interface that identifies potentially important concepts in documents to bridge the gaps between information scattered across webpages about the same concepts. This has the potential of allowing users to fluidly capture and associate information from across webpages and their notes, and efficiently build up their understanding of the information space it make informed decisions as they examine and collect information from multiple online sources.

