


With the proliferation of Web content, foraging and consuming information online has became how many people make sense of the world today. From making purchase decisions based on product reviews to learning about a new topic by reading tutorials and how-to articles, people now have instant access to an enormous bazaar of online information generated others with different personal preferences, backgrounds, and domain knowledge. While having access to this rich repository of diverse perspectives has the potential to empower consumers and learners to understand their choices thoroughly and make better decisions for themselves and without being overly influenced by marketing and branding \cite{de2015navigating}, people are often unable to effectively utilize the full potential of online data, struggling to manage, collect, and organize information that are scattered across multiple online sources into useful structures.

Anecdotal evidence for this need can also been seen in the rise of aggregation-based sites such as Meta-critics and Wirecutter that try to ameliorate this issue by showing average review ratings and compiling meta-analysis of evidences and opinions scattered across multiple sources. However, aggregation-based approaches can be difficult to scale to support scenarios beyond purchase decisions that have less well-defined options and criteria, such as figuring out ``how to grow better tomatoes'' or learning ``how to make bread.''
Further, aggregation-based approaches fails to take into account the personal context of the users and their goals. For example, product meta-analysis websites like Meta-critics scores other review websites to generate weighted average ratings, and Wirecutter recommends products that they determined "best for most people". Recent research has also shown that sifting through individual reviews still plays an important role in online purchase decision making even when average ratings were presented \cite{mudambi2010research,gan2012helpfulness}.

Fundamentally, exploratory searchers do not start with clear ideas of what they are looking for in mind \cite{mar2006exp}, and have to iteratively discover and refine their criteria and potential options as they learn more about the space of information. While in most cases there an abundance of useful information online, this foraging process can be prohibitively costly for the users to fully benefit from them \cite{pirolli1999information,pirolli2005sensemaking}, because it can be difficult to keep track of the different options and potentially conflicting evidence that were scattered across different webpages. I my thesis work, I focus on bridging the gap between relevant information scattered across the Web to support different stages of the exploratory search process, from providing an overview of the information space in early stages, to allowing users to incrementally develop their criteria based on data, to collecting and structuring information towards making decisions.

\section{Discovering options and aggregating evidence}
    -- KA + Alloy (+Revolt) + SearchScape
    
\section{Foraging while Developing Personal Goals and Criteria}
    -- SearchLens (+ Tabs) + Highlighting (+ Fusion)
